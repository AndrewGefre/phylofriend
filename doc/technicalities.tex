\section{Technicalities}

\subsection{Source Code Documentation}

To access the source code documentation point your
web browser to:

\begin{itemize}
\item \href{http://godoc.org/code.google.com/p/phylofriend}{http://godoc.org/code.google.com/p/phylofriend}
\end{itemize}

If you want to modify the source code it is best to
use \emph{godoc} locally on your computer.

\begin{enumerate}
\item If \emph{godoc} is not yet installed install it by typing\\
	\texttt{sudo apt-get install golang-go.tools}\\
	(Linux Mint)
\item Start the documentation web server with\\
	\texttt{godoc -http=:6060}
\item Point your web browser to \texttt{localhost:6060}.
\item Click on \emph{Packages} and search for \emph{phylofriend}.
\end{enumerate}

This will give you a nice overview of the internal program
documentation. You can also click on function names to browse
the source code.


\subsection{Mutation Model}

The two basic mutation models are the infinite alleles model
and the stepwise mutation model as explained by Bruce Walsh\cite{Wal02}.
Phylofriend uses a hybrid mutation model. Most markers are
calculated using the stepwise model, palindromic markers are
calculated as described in \cite{Can14}.

As the method for calculating the genetic distance is likely
to change with time please look at the internal program
documentation if you need more details.


\subsection{CSV Input Format}

Example:

\begin{verbatim}
id1,"Dirk Struve",Germany,R1b-CTS4528,13,24,14,11,11-14,12,...
id2,"Pyl. O. Friend",Germany,R1b-CTS4528,13,24,14,11,11-14,...
\end{verbatim}

When importing a file in comma separated values format the
first column must contain IDs. An arbitrary number of columns
containing custom information may follow. The last columns
must contain at least 12 Y-STR values in Family Tree DNA order.
Rows containing comments are allowed.

Phylofriend will always try to parse the file as best as
it can.


\subsection{Text Format}

Example:

\begin{verbatim}
Dirk_Struv	13	24	14	11	11	14	12	12	12	12	14	28
Pyl._O._Fr	13	24	14	11	11	14	12	12	12	12	14	28
\end{verbatim}

The text format is a simplified format intended for easy
parsing and to work well with other programs. For compatibility
reasons the first column is exactly 10 characters long and
contains only non Unicode characters. Spaces are transformed into
underscores. The following columns contain Y-STR values
separated by tabs.


\subsection{PHYLIP Format}

Example:

\begin{verbatim}
2
Dirk_Struv	0	0
Pyl._O._Fr	0	0
\end{verbatim}

The first line contains the number of entries. An entry
line contains an ID that is 10 characters long and contains
only non Unicode characters. Spaces are transformed into
underscores. The columns containing genetic distances
are separated by tabs. For readability reasons
Phylofriend writes only integers. If you need more precision
you can scale the distance by using the \emph{cal} option.







