\section{Examples}

\subsection{Create a Phylogenetic Tree}

\begin{enumerate}
\item Copy persons' data from a Family Tree DNA project website into
	a spreadsheet. If the Y-STR values do not appear properly try
	inserting them into the spreadsheet as unformatted text.
\item Save the spreadsheet in CSV (comma separated values)
	format, for example \emph{persons.csv}.
\item Start a terminal or command line interpreter and go
	to the directory where you stored \emph{persons.csv}.
\item Create a matrix of genetic distances by typing\\
	\texttt{phylofriend -personsin persons.csv -phylipout infile}
\item Use the PHYLIP program to create a tree in Newick format
	with\\
	\texttt{/usr/lib/phylip/bin/kitsch}\\
	You will need to answer some questions. Usually the
	default values are good enough. The results will be
	two text files, one named \emph{outtree} which contains
	the tree in Newick format and another one named
	\emph{outfile} which contains a more human readable
	description.
\item Create an image of the tree by typing\\
	\texttt{/usr/lib/phylip/bin/drawgram}\\
	Use \emph{outtree} as the input file name.
	The resulting tree will be stored in a file named 
	\emph{plotfile}.

	A nice alternative to visualize the tree is the use of the
	\href{http://www.trex.uqam.ca/index.php?action=newick&project=trex}
	{Trex}\cite{Trex}
	webserver. You can copy the contents of the file \emph{outtree}
	into the Trex window.
\end{enumerate}

If you do not specify a file containing mutation rates
Phylofriend will use average 37 marker values as default.
They can be found in \cite{Kly12}.


\subsection{Pimp Your Tree with Nicer Labels}

By default Phylofriend assumes that your persons input file's
first column contains a list of IDs. This is usually a Family
Tree DNA Kit number. The resulting tree is hard to read. Many
projects keep names in another column. You can access this 
column with the \emph{namescol} option. Suppose your second
column contains names. You can create a distance matrix with
names instead of IDs by typing

\noindent\texttt{phylofriend -personsin persons.csv -namescol 2\\
-phylipout infile}

Due to compatibility issues with other programs the names
should be 10 characters long and may not contain Unicode
characters. Phylofriend will apply a name transformation
but the result is sometimes a bit strange.

You can also use the \emph{namescol} option to create trees
that contain the origins of people or the haplogroups. Although
I strongly recommend to build trees only from people who
belong to the same haplogroup this is sometimes useful if
you want to know if different haplogroups are close on their
Y-STR values.

If you want to publish your tree you will often need to
protect the privacy of the members. This is what the
\emph{anonymize} option is for. By typing

\noindent\texttt{phylofriend -personsin persons.csv -phylipout infile -anonymize}

you will get a distance matrix where the names are replaced
by numbers.


\subsection{Use a Specific Set of Mutation Rates}

The first example uses Phylofriend's build in mutation rates
which are average values for the standard 37 marker test.
Phylofriend supports the use of arbitrary mutation rates by
the \emph{mrfile} option. The \emph{phylofriend/mutationrates}
directory contains some files with mutation rates. The average
mutation rates where taken from \cite{Kly12}.
If you like to compare on 67 markers you can use

\noindent\texttt{phylofriend -personsin persons.csv -phylipout infile\\
-mrin code.google.com/p/phylofriend/mutationrates/67-average.txt}


\subsection{Calibrate Your Data}

Mutation rates depend on the method applied to calculate
genetic distances and the sample populations used. Mutations
themselves occur by coincidence. Average mutation rates
often yield acceptable results but in most cases you will
have to calibrate your data especially if you want to
calculate in years.

Phylofriend provides two options for data calibration:
\emph{gendist}, the generation distance in years and
\emph{cal} an additional calibration factor. Internally
they are just multiplied together but using two separate
factors seems more convenient for typical use cases.

The default value for the generation distance is 25 years.
For time spans over the last few hundred years a generation
distance of 30 years often yields better results. This can
be done by

\noindent\texttt{phylofriend -personsin persons.csv -phylipout infile\\
-gendist 30}

It is often difficult to calibrate data because you need a
reliable paper trail or a well defined historic event. If
you are lacking both you can try to apply Klyosov's statistical
method\cite{Kly09}. For large enough sample sizes this will
effectively reduce the statistical error but you will still
be left with an unknown systematical one.


\subsection{Count Mutations}

The \emph{phylofriend/mutationrates} directory contains
sets of mutation rates where all markers are set to 1.
This makes mutation counting easy but you will need an
additional little trick. Internally Phylofriend uses
average values. So it is possible to compare persons who
have tested on different sets of markers.

If you want to count mutations for example on a 37 marker
scale, you must multiply Phylofriend's internal results with
37. The easiest way to archive this is by misusing the
\emph{gendist} option. 

\noindent\texttt{phylofriend -personsin persons.csv\\
-mrin code.google.com/p/phylofriend/mutationrates/37-1.txt\\
-phylipout distancecount.txt -gendist 37}


\subsection{Extract Data from a Spreadsheet}

Spreadsheet data is often uncomfortable to handle, especially
if you want to write your own program and need to parse it.
For this purpose Phylofriend supplies the \emph{txtout}
option. It writes data to a text file in simplified form.

The easiest way to use it is

\noindent\texttt{phylofriend -personsin persons.csv -txtout persons.txt}

This extracts the data from \emph{persons.csv} and writes
it to \emph{persons.txt}. The first column of \emph{persons.txt}
contains the first column found in \emph{persons.csv}, usually
a set of IDs. The following columns contain the Y-STR values.
All columns are separated by tabs.

If you want to use another column as the first column you can
use the \emph{namescol} option:

\noindent\texttt{phylofriend -personsin persons.csv -txtout persons.txt\\
-namescol 2}

This extracts the second column from \emph{persons.csv} and
writes it to the first column of \emph{persons.txt}.

When using the \emph{txtout} option you will need to specify
how many Y-STR values are written to the text file. This is
done by \emph{nvalues}. Phylofriend will write the same number
of Y-STR values to each line. Missing values are written as
0, larger sets of Y-STR values are truncated. This is how to
write a full set of 111 markers:

\noindent\texttt{phylofriend -personsin persons.csv -txtout persons.txt\\
-nvalues 111}
















