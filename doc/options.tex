\section{Command Line Options}

Command line options may be given in arbitrary order.

\begin{description}
\item[-help] Prints available program options.
\item[-personsin] Filename of the file containing the persons' Y-STR values.
	This may be in CSV (comma separated values) or text format.
\item[-namescol] Number of the column that contains the names
	when reading CSV files.
\item[-mrin] Filename of the mutation rates to use.
\item[-anonymize] If this is true persons' names are replaced by numbers.
\item[-modal] Creates modal haplotype.
\item[-phylipout] Filename for the distance matrix that can be fed into
	the PHYLIP\cite{Phylip} program.
\item[-mrout] Filename for the output of the currently used mutation rates.
\item[-txtout] Filename for text output of persons and Y-STR values.
\item[-nvalues] Number of Y-STR values to write to the text output file.
\item[-gendist] Generation distance.
\item[-cal] Calibration factor.
\item[-reduce] Reduces the number of persons by the given factor.
\end{description}

