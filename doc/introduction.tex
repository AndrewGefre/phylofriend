\section{Introduction}

Phylofriend's main purpose is to calculate genetic distances from
Y-DNA data. The results can be used as input for the
PHYLIP\cite{Phylip} program to create phylogenetic trees.

When I started creating phylogenetic trees I often found
myself in a difficult position. As a Linux user I was missing
some of the tools available under Windows. So I started
to write this program to fill in the gaps and make myself
comfortable again.

This does not mean that you can not use Phylofriend when
working under Windows or the Mac. But currently there is
no binary distribution available and you will probably face
a hard time installing Phylofriend and the associated programs.
So I only recommend this if you are an experienced user.

Phylofriend has some nice features. It can be used

\begin{itemize}
\item to create phylogenetic trees using the
	\href{http://evolution.genetics.washington.edu/phylip.html}{PHYLIP}\cite{Phylip}
	program. Y-STR values from Family Tree DNA projects can
	easily be imported.
\item as a programming library. Phylofriend is written in
	Google's \href{http://golang.org/}{Go} programming
	language. This language is not only suited to solve
	Google's large scale programming problems. It is also
	an excellent tool for part time programmers who have
	to concentrate on their projects (often students).
\item to extract Y-DNA data from Family Tree DNA projects
	and convert it into simpler text files that are
	better suited for further processing.
\item to automate phylogenetic tree creation. Phylofriend
	is a command line tool and this scares many people
	away. But if you have to repeat the same tasks over
	and over again you will eventually start to write some
	scripts and this is where command line tools come in
	handy.
\end{itemize}

I hope this program will be useful. Have a good time!

\vspace{1em} Dirk



